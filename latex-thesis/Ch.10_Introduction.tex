\chapter{Introduction\label{intro}}
% TODO:
    % Konteksti (kuka tarvitsee, mitä, miksi ongelma)
    % Tutkimuskysymyksen ja -menetelmän asettaminen, tulosten yhteenveto
    % Lukujen roolitus

\comment{testi}

Python is a general purpose programming language that is in widespread use. Recently there have been developments in making use of gradual typing via type annotations and inference to aid in Python development.

Python's growing market share, along with growing popularity of typing solutions for Python, seemingly contains a contradiction: Why adopt a dynamic language, if you are going to statically type the code anyway?

In statically typed languages the source code is typed at compile time, with type hints and type inference. A dynamic language assigns types at runtime. This affects the way programmers interact with type errors and can shift the focus point of correctness guarantees from compilation time to  runtime.

The choice between a statically or dynamically typed language can have a modest but statistically significant effect on code quality \cite{ray_codequality_2014}.
However in a reproduction study \cite{codequality_reproudction_2019} the result could not be reproduced. Analysing empirical result differences between different programming languages is difficult.
\comment{TODO: selitä miksi nää kaksi paperia on hyvä sample tän kentän tilanteesta, tai kaiva metastudy}

Type-annotated Python can be described as a gradually typed language. Gradual typing is a mix of dynamic and static typing. Applying gradual typing to existing dynamic languages, such as using TypeScript over JavaScript can provide statistically significant improvements in code quality. %cite typescript sources

This thesis provides technical background information for Python, its type system, and type checking solutions for Python. Then we analyse how, why, and to what effect type annotations are being used for in Python.

\comment{TODO: tulosten yhteenveto}