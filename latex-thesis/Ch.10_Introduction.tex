\chapter{Introduction\label{intro}}
% TODO:
    % Konteksti (kuka tarvitsee, mitä, miksi ongelma)
    % Tutkimuskysymyksen ja -menetelmän asettaminen, tulosten yhteenveto
    % Lukujen roolitus


Python is a general-purpose dynamically typed programming language. It has been growing fast in popularity since early 2010s.
% TODO: viittaa SO developer surveyhyn
In general dynamic programming languages have been gaining market share. In certain problems dynamic languages provide rapid prototyping capabilities and work better for newcomers. % An Empirical Comparison of Seven Programming Languages (Koitin kaivaa, niiden alkuperäislähde tolle on heikko / statee että tulokset ei yleistyisi...)

In 2014 Python introduced PEP-484 to support type annotations, also known as type hints. \cite{pep_484} The solution specifies that by default the annotations will be available at runtime. It does not specify a way to utilize them. Various type checkers have been developed. %citation needed

These type checkers provide ways to statically analyse Python source code to find type errors. Type errors can help developers find and possibly fix software defects. \cite{khan_empirical_2022} 

This paper provides an overview of ways that Python source code can be type annotated and type checked. Then we investigate the costs and benefits of type annotating Python.

