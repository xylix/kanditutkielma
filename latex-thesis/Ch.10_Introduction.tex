\chapter{Introduction\label{intro}}
% TODO:
    % Konteksti (kuka tarvitsee, mitä, miksi ongelma)
    % Tutkimuskysymyksen ja -menetelmän asettaminen, tulosten yhteenveto
    % Lukujen roolitus

Python is a general purpose programming language that is in widespread use. Since 2015 there has been support for optional type hinting, which enables adding additional type information to the source code to aid development.

Python's growing market share, along with growing popularity of typing solutions for Python, seemingly contain a contradiction: Why adopt a dynamic language, if you are going to statically type the code anyway?

In statically typed languages, the source code is assigned types at compile time, with type hints and type inference. A dynamic language assigns types at runtime. This affects the way programmers interact with type errors and shifts the focus point of guarantees from compilation time to runtime.

The choice between a statically or dynamically typed language can have a modest but statistically significant effect on code quality \cite{nanz_comparative_2015, ray_codequality_2014}. However in a reproduction study \cite{codequality_reproudction_2019} the result from \cite{ray_codequality_2014} could not be reproduced. Authors concluded that analysing quantitative differences from large-code samples from different programming languages can easily contain statistical errors that make the results misleading or overconfident.

Python with the optional types in use (type-annotated Python) can be described as a gradually typed language. Gradual typing is a mix of dynamic and static typing. Applying gradual typing to existing dynamic languages, such as using TypeScript over JavaScript, can provide statistically significant improvements in detecting bugs \cite{gao_to_type_or_not_2017}. Similar results have been found for type-annotated Python \cite{khan_empirical_2022, rak-amnouykit_taleoftwo_2020}.

Chapter \ref{background} provides technical background information for type systems, Python, its type system, and type checking solutions for Python. Chapter \ref{related_work} analyses how type annotations are researched, how they are used, and to what effect type annotations are being used in Python. Then discussion \ref{discussion} presents hypothesises on why use Python and why type annotate it. Chapter \ref{conclusions} concludes the researched knowledge into answers.

\comment{TODO: before turning in: Ensure the structure description is correct. Write it to flow better.}