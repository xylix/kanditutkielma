\chapter{Introduction\label{intro}}
% TODO:
    % Konteksti (kuka tarvitsee, mitä, miksi ongelma)
    % Tutkimuskysymyksen ja -menetelmän asettaminen, tulosten yhteenveto
    % Lukujen roolitus

Python is a general purpose programming language that is in widespread use. Since 2015 there has been support for optional type annotations (type hinting), which enables adding additional type information to the source code to aid development.

Python's growing market share, along with the growing popularity of typing solutions for Python, seemingly contain a contradiction: Why adopt a dynamic language, if you are going to statically type the code anyway?

In \emph{statically typed} languages, the source code is assigned types at compile time, with type annotations and type inference. A \emph{dynamically typed} language assigns types at runtime. This affects the way programmers interact with type errors and shifts the focus point of guarantees from compilation time to runtime.

% Tälle static - dynamic rajalle tulee toistoa myöhemmissä chaptereissä
The choice between a statically or dynamically typed language can have a modest but statistically significant effect on code quality \cite{nanz_comparative_2015, ray_codequality_2014}. However in a reproduction study \cite{codequality_reproudction_2019} the result from \cite{ray_codequality_2014} could not be reproduced. The authors concluded that analysing quantitative differences from large-code samples from different programming languages can easily contain statistical errors that make the results misleading or overconfident.

Python with the optional types in use (\emph{typed Python}) can be described as a gradually typed language. \emph{Gradual typing} is a mix of dynamic and static typing. Applying gradual typing to existing dynamic languages, such as using TypeScript over JavaScript, can provide statistically significant improvements in detecting software defects (bugs) \cite{gao_to_type_or_not_2017}. Similar results have been found for typed Python \cite{khan_empirical_2022, rak-amnouykit_taleoftwo_2020}.

This thesis aims to provide an overview for Pythons type system and type annotations, and their usage and benefits. Chapter \ref{background} provides technical background for type systems, Python, its type system, and type-checking solutions for Python. Chapter \ref{related_work} analyses the research and practical applications of type annotations, and their implementations and impact. Then discussion \ref{discussion} presents motivations and reasons for adopting Python and type annotations. Chapter \ref{conclusions} proceeds to synthesize the research insights into answers.