\chapter{Introduction\label{intro}}
% TODO:
    % Konteksti (kuka tarvitsee, mitä, miksi ongelma)
    % Tutkimuskysymyksen ja -menetelmän asettaminen, tulosten yhteenveto
    % Lukujen roolitus

\comment{testi}

Python is a general purpose programming language that is in widespread use. Recently there have been developments in making use of gradual typing via type annotations and inference to aid in Python development.

Python's growing market share, along with growing popularity of typing solutions for Python, seemingly contains a contradiction: Why adopt a dynamic language, if you are going to statically type the code anyway?


It has been found that the choice between a statically or dynamically typed language can have a modest but statistically significant effect on code quality \cite{ray_codequality_2014}.
In a reproduction study \cite{codequality_reproudction_2019} the result could not be reproduced.
\comment{TODO: selitä miksi nää kaksi paperia on hyvä sample tän kentän tilanteesta, tai kaiva metastudy}

However in gradually typing dynamic languages, various sources find that using Typescript over JavaScript can provide statistically significant improvements in code quality. %cite typescript sources


\comment{TODO: fix this article paremmaksi muotoiluksi}
This thesis provides a background in Python, its type system, and type checking solutions available for Python. Then we analyse how, why, and to what effect type annotations are being used for in Python.

\comment{TODO: core research question:}
\comment{TODO: tulosten yhteenveto}


\comment{TODO: old stuff, perhaps delete all:}


% Python is a general-purpose dynamically typed programming language. It has been growing fast in popularity since early 2010s.
% TODO: viittaa SO developer surveyhyn
% In general dynamic programming languages have been gaining market share. In certain problems dynamic languages provide rapid prototyping capabilities and work better for newcomers. % An Empirical Comparison of Seven Programming Languages (Koitin kaivaa, niiden alkuperäislähde tolle on heikko / statee että tulokset ei yleistyisi...)

% In 2014 Python introduced PEP-484 to support type annotations, also known as type hints. \cite{pep_484} The solution specifies that by default the annotations will be available at runtime. It does not specify a way to utilize them. Various type checkers have been developed. %citation needed

% These type checkers provide ways to statically analyse Python source code to find type errors. Type errors can help developers find and possibly fix software defects. \cite{khan_empirical_2022} 

% This paper provides an overview of ways that Python source code can be type annotated and type checked. Then we investigate the costs and benefits of type annotating Python.

