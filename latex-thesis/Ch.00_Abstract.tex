% \begin{abstract}{finnish}

% Tämä dokumentti on tarkoitettu Helsingin yliopiston tietojenkäsittelytieteen osaston opin\-näyt\-teiden ja harjoitustöiden ulkoasun ohjeeksi ja mallipohjaksi. Ohje soveltuu kanditutkielmiin, ohjelmistotuotantoprojekteihin, seminaareihin ja maisterintutkielmiin. Tämän ohjeen lisäksi on seurattava niitä ohjeita, jotka opastavat valitsemaan kuhunkin osioon tieteellisesti kiinnostavaa, syvällisesti pohdittua sisältöä.


% Työn aihe luokitellaan  
% ACM Computing Classification System (CCS) mukaisesti, 
% ks.\ \url{https://dl.acm.org/ccs}. 
% Käytä muutamaa termipolkua (1--3), jotka alkavat juuritermistä ja joissa polun tarkentuvat luokat erotetaan toisistaan oikealle osoittavalla nuolella.

% \end{abstract}

\begin{otherlanguage}{english}
\begin{abstract}

This thesis is a literature review on Python type annotations, their use and benefits. Background on Python, type systems, type annotations and type checking is provided. Then the relevant literature is discussed and analysed.

Python is a widely utilized general purpose programming language. It's type system is strong and dynamic. In 2015 support for optional type annotations was implemented.

Python is well-known as simple, concise and readable. Type annotations may sometimes come in conflict with these. However the gradual nature of optional typing enables developers to choose and pick where to utilize type annotations. Type annotations have had a significant rise in popularity from 2015 to 2022. Still only a minority of Python source code utilizes them. Python type inference and checking has been integrated into code editors, and this helps developers work with both typed and untyped source code.

Existing research has determined that it is possible to find approximately 15\% software defects (bugs) by utilizing type annotations. Type annotations can be utilized to document public interfaces and to ensure correctness in desired areas.

\comment{
Finally, specify 1--3 ACM Computing Classification System (CCS) topics, as per \url{https://dl.acm.org/ccs}.
Each topic is specified with one path, as shown in the example below, and elements of the path separated with an arrow.
Emphasis of each element individually can be indicated
by the use of bold face for high importance or italics for intermediate
level.}

\end{abstract}
\end{otherlanguage}
