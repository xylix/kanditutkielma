% \begin{abstract}{finnish}

% Tämä dokumentti on tarkoitettu Helsingin yliopiston tietojenkäsittelytieteen osaston opin\-näyt\-teiden ja harjoitustöiden ulkoasun ohjeeksi ja mallipohjaksi. Ohje soveltuu kanditutkielmiin, ohjelmistotuotantoprojekteihin, seminaareihin ja maisterintutkielmiin. Tämän ohjeen lisäksi on seurattava niitä ohjeita, jotka opastavat valitsemaan kuhunkin osioon tieteellisesti kiinnostavaa, syvällisesti pohdittua sisältöä.


% Työn aihe luokitellaan  
% ACM Computing Classification System (CCS) mukaisesti, 
% ks.\ \url{https://dl.acm.org/ccs}. 
% Käytä muutamaa termipolkua (1--3), jotka alkavat juuritermistä ja joissa polun tarkentuvat luokat erotetaan toisistaan oikealle osoittavalla nuolella.

% \end{abstract}

\begin{otherlanguage}{english}
\begin{abstract}

This thesis is a literature review on Python type annotations, their use and benefits. Background on Python, type systems, type annotations and type checking is provided. Then the relevant literature is discussed and analysed. The target audience is users and researchers of Python.

Python is a widely utilised general purpose programming language. Its type system is strong and dynamic. In 2015 support for optional type annotations was implemented. Type annotation's effectiveness in improving program correctness is the main motivation for this study.

Python is well-known as simple, concise and readable. Type annotations may potentially conflict with these properties; however, the gradual and optional features of type annotations let developers implement type annotations where they provide the most value.

Type annotations have risen in popularity from 2015 to 2022. Nevertheless, only a minority of Python source code utilises type annotations. Python type inference and checking has been integrated into code editors, and this helps developers work with both typed and untyped source code.

Existing research has determined that it is possible to find approximately 11\% software defects (bugs) by utilizing type annotations. Type annotations can be utilised to document public interfaces and to improve correctness in selected parts of a software.

\end{abstract}
\end{otherlanguage}
