\chapter{Background\label{background}}

\section{Type systems}

Type systems are tools that enable abstract organization and categorization of data and related tools \cite{programming_langs}. They allow the programming language to give guarantees about data structures, they enable varying behaviour based on the specified type, and they can do work as inferred, controllable comments.

Effectively all modern high level programming languages make use of type systems. The systems vary in features, granularity, constraints and inference logic. Languages can be classified to static, dynamic or gradual by how their type system is applied, and to strong and weak languages based on the strength of the type system guarantees.

\begin{table}[h!]
    \centering
    \begin{tabular}{@{}lcl@{}}
    \toprule
    Type system features       & Languages      \\ \midrule
    Static          & C,  Java           \\
    Dynamic         & Python, JavaScript \\ 
    Gradual         & TypeScript, Typed Python \\ \midrule
    Strong          & Python, Java \\
    Weak            & C, JavaScript, TypeScript  \\ \bottomrule
    \end{tabular}
    \caption{Examples of programming languages with varying type system features.}
    \label{table:1}
\end{table}
% Could include c++ but it's strong - weak distinction may be confusing. Cut Java and Rust to make the table more conscise.

\subsection{The strong -- weak distinction}
Type errors are raised when an operation is called with incorrect (runtime) types. In Python 3.12.6 executing {\tt 1 + "a"} yields the error {\tt "TypeError: unsupported operand type(s) for +: 'int' and 'str'" }. Programming languages vary in implicit conversions between types when an operation is applied to differently typed arguments. 

A reason to avoid implicit conversion is to ensure a higher level of correctness, avoiding avoid unchecked runtime type errors which can lead to undetected erroneous behaviour. A language that aims for no unchecked runtime type errors has a strong type system \cite{cardelli_typeful_1989}. This means that runtime type errors must panic or cause checkable exceptions. Languages that allow unchecked runtime type errors and implicit runtime conversion between most types have weak type systems.

\subsection{Static typing}
Static typing requires a user to specify types at compile time, or for the compiler to have the ability to correctly infer types for all values. Tools for static languages can provide better suggestions and error detection, since more data is available for introspection. Compilers can also perform more effective, sometimes zero-cost optimizations for code based on type information. It is commonly believed that statical languages have better performance when optimized, and programming with them results in more correct programs. The first is often true, but can vary case-by-case. \cite{nanz_comparative_2015} The second is hard to analyse empirically \cite{codequality_reproudction_2019}, perhaps due to the fact that many other qualities affect the correctness of written programs. 

\subsection{Dynamic typing}
In a dynamically typed language the final assignments of types happens at runtime. There can even be situations where unknown data is received into the program and parsed into a type in runtime. Dynamic languages feature lightweight syntax that can make it faster to get going \cite{di_grazia_evolution_2022}. Dynamic languages are often taught as first programming languages, by initially leaving off the complexity of type systems.

% secret TODO: the evolution cite is not good quality here

\subsection{Gradual typing}
Gradual typing can be thought of as a superset of static and dynamic typing \cite{siek_refined_gradual_2015}. In a gradually typed language one can write a statically typed program, a dynamically typed program, or various in-between programs. The unique advantage of gradual typing is the ability to type important sections of a larger program, and evolve these typed sections and their proportion gradually \cite{siek_refined_gradual_2015}. Gradual typing is often implemented as a superset of an existing programming language.

% secret TODO: could explain type inference abstractly here
% secret TODO: could also explain type errors?

\section{Python}
Python is a high-level interpreted programming language whose syntax uses significant whitespace for conciseness. It has a strong and dynamical type system. Significant usage domains include scientific computing, machine learning, web backend development and scripting. The concise syntax and capability to make programs executable quickly have been significant advantages in growing market share from stricter, compiled languages such as Java, C\# and C++.


\section{Types in Python}

Type annotated Python has two type systems which have different scopes:
\begin{enumerate}
    \item The runtime type system is semantically and behaviourally relevant. The type of a Python object at runtime contains data on its values shape and possible behaviour for the object. The objects available methods are interpreted based on its runtime type.
    \item The optional annotation system was added in 2014 after Python developers drafted Python Enhancement Proposal 484 - Type hints \cite{pep_484}. Ability to do improved static analysis, and possibility for improved refactoring, runtime type checks, and code generation were the motivation, with static analysis documented as the most important one. A prototypal type checker and the ability to add type metadata through a generic mechanism already existed, but this proposal standardized the way to annotate Python code with type data, enabling improved tooling development across the field.
\end{enumerate}

Enforcement and performing any static analysis on the annotations described in 2) is completely optional and up to the developer \cite{python_typing}.

\subsection{Type annotations}
In this thesis we focus on discussing the annotation-based higher level gradual type system. Python's implementation of type annotations does not affect program behaviour like runtime types do. In theory one can run type annotated Python code without ever executing a type check, and without making use of the annotated type data. This follows the \emph{gradual guarantee} \cite{siek_refined_gradual_2015}, where a type annotated and non-annotated program should only differ in their type annotations. Type annotations are denoted via {\tt :} for variables and function arguments and {\tt ->} for return values, as can be seen in \ref{typed_code}.


\begin{figure}[t]
    \centering
     \begin{minipage}{.45\textwidth}
        \centering
        \begin{lstlisting}[linewidth=\textwidth]
def f(i, j):

    sum = i + j
    return sum % 2 == 0
        \end{lstlisting}
        \caption{Without type annotation}
        \label{untyped_code}
    \end{minipage}%
    \begin{minipage}{.45\textwidth}
        \centering
        \begin{lstlisting}[linewidth=\textwidth]
def f(i: int, j: int) -> int:
    sum: int = i + j
    return sum % 2 == 0
        \end{lstlisting}
        \caption{Type annotated}
        \label{typed_code}
    \end{minipage}
\end{figure}

\subsection{Python type checking}

% secret TODO: Do a comparison of the following tools in related work / further analysis
As of 2024 there are three widely adopted tools for Python type checking: Mypy, Pyright, PyCharm. There are also general linters and other code quality tools that type check, but these three are the most prevalent.

Mypy by Python foundation is a common tool to run in Continuous Integration environments to do type checking across tests. It has a command line interface and can be utilized through Language Servers in code editors. 

Microsoft's Visual Studio Code is the most popular editor for Python programming \cite{python_software_foundation_jetbrains_sro_python_nodate}. It utilizes Pyright for type checking by default, in-line type errors.

PyCharm is a commercial Python IDE developed by JetBrains s.r.o. It is the second most popular Python editor and the most popular full IDE. PyCharm has its own integrated type checker. PyCharm, being an IDE, is rarely utilized outside of the developers machine. 

It is common for developers to develop code with VS Code or PyCharm and then run {\tt mypy } in a continuous integration environment, providing a kind double checking with slightly different prioritisations affecting what situations each checker considers errors.

% It is plausible that in many projects the configuration between the two does not match 1:1, leading to situations where a programmer could commit locally type checking code that fails in CI, or another programmer could pull passing code from CI that does not pass locally. This phenomena is not common in other programming languages, where often a single type checker or the programming languages own compiler has a monopoly.

\subsection{Type inference}

Mypy, Pyright and PyCharm all support type inference \cite{jetbrains_type_hinting_pycharm, mypy_type_inference, pyright_type_inference}. This means that they can \emph{infer} what types a variable is from the values assigned to it. They also do inference for function return values from the values that are returned. Type inference enables some benefits from type checking even when code is minimally typed, and is commonly used for editor features such as autocompletion and detecting type errors before runtime.

\begin{figure}[t]
    \centering
    \begin{minipage}{0.5\textwidth}
        \centering
        \begin{lstlisting}[linewidth=\textwidth]
def f() -> bool:
    num = 0
    return num
        \end{lstlisting}
    \caption{Inferred type error}
    \label{inferred_type_error}
    \end{minipage}
\end{figure}

Mypy 1.13.0 returns the following when checking the source code in \ref{inferred_type_error}: \begin{verbatim}
type_inference_example.py:3: error: 
Incompatible return value type (got "int", expected "bool") [return-value]
\end{verbatim}

The error indicates that on line 3 the type checking rule {\tt [return-value]} was violated. It also indicates that the value that is being returned is type {\tt int} indicating that {\tt num}'s type was correctly inferred.