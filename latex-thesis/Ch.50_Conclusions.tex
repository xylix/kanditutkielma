\chapter{Conclusions\label{conclusions}}

\section{Key findings}
Python is adopted for a wide variety of reasons. Statical type hints are used for reasons that sometimes overlap with these, but the reasons are not always conflicting.

Type annotations are being adopted more widely. Using type annotations and type checking Python code helps developers document APIs and find errors before running the code. Type checking has been made easier by the adoption of LSP-based tooling. \comment{TODO: Expand "Why type check python" section and then conclude from the results here.}

Type annotations enable a gradual approach for finding bugs earlier and documenting APIs in ways that automated tooling can make use of. They allow developers to utilize Pythons core strengths while improving on shortcomings that have historically widely applied to dynamic scripting languages.

\comment{TODO Is this useful to state?:}
The research can not yet conclude a satisfying answer to the question "Where, and how one utilize Python type annotations?"

There are avenues for future research on utilizing Python type annotations for more benefit, comparing type checkers, and figuring out what the adoption rate and quality of type hints is now, in 2024. 

\comment{TODO: Edit for better text quality.} Research to use the type annotations for performance and correctness benefits in runtime, and popularity of existing solutions could be useful. In general considering the size of Python there exist a lot of untapped opportunities for Python type related research.