\chapter{Conclusions\label{conclusions}}

Python is widely adopted for various reasons, including ease of use, ease of learning, and the wide availability of packages. In comparison to other programming languages it lands at a useful compromise on readability, expressivity and conciseness.

Utilising type annotations types makes Python into a gradually typed programming language, with increased readability, self-documentation and correctness, with costs in additional syntactic noise and additional development effort. This has led to type annotations being adopted more widely \cite{jin_where_to_start_2021, khan_empirical_2022}. Type checking has been made easier by the popularity of editors that do it by default. This assists developers in finding defects and utilising type information better. 

Typed Python allows developers to utilize Python's core strengths while improving on shortcomings that have historically widely applied to dynamic scripting languages.

Python projects should adopt type annotations when they want to spend development efforts on improving the maintainability, correctness, and static analysis of their code base. Intentions on coverage, specifically which internal and public APIs should end up covered, should be set  \cite{jin_where_to_start_2021}. Some time should be spent on estimating where correctness and analysis improvements provide the most benefit. Often this would be parts of the software where type annotating is relatively easy, and which often cause bugs. When type annotating proves difficult it can be an indicator that the program structure is complicated, and could benefit from refactoring.

Promising avenues for future research include: comparing type checkers, especially PyCharm and Pyright, optimal adoption of type annotations (including instructions for developers), and comparing type annotations to other quality assurance tools such as automated testing to determine where correctness investments provide the most benefit. 

% Avenues for future research include: 1) utilizing Python type annotation most effectively, 2) runtime correctness and performance benefits from type annotations 3) comparing type checkers, 4) studying the adoption rate and quality of type annotations since 2022, 5) instructions on how to best utilize type annotations, 6) comparing type annotations to other software quality assurance tools, such as automated testing, and determining optimal correctness effort - benefit tradeoffs.


% Considering size of Python there seem to be lots of untapped avenues for future research.