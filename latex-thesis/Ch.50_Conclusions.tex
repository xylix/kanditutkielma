\chapter{Conclusions\label{conclusions}}
\comment{TODO: Johdattele discussionista tähän}

\section{Key findings}
Python is adopted for a wide variety of reasons. Statical type hints are used for reasons that sometimes overlap with these, but the reasons are not always conflicting.

Type annotations are being adopted more widely. Using type annotations and type checking Python code helps developers document APIs and find errors before running the code. Type checking has been made easier by the adoption of LSP-based tooling 

Type annotations enable a gradual approach for finding bugs earlier and documenting APIs in ways that automated tooling can make use of. They allow developers to utilize Pythons core strengths while improving on shortcomings that have historically widely applied to dynamic scripting languages.

There are avenues for future research on utilizing Python type annotations to highest benefits, comparing type checkers, and figuring out what the adoption rate and quality of type hints is now, in 2024. 