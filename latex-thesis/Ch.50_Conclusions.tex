\chapter{Conclusions\label{conclusions}}

Python is adopted for a wide variety of reasons, including ease of use, ease of learning, and the wide availability of packages. In comparison to other programming languages it lands at an useful compromise on readability, expressivity and conciseness.

Utilizing type annotations types makes Python into a gradually typed programming language, with increased readability, self-documentation and correctness, with costs in additional syntactic noise and additional development effort. This has lead to type annotations being adopted more widely \cite{jin_where_to_start_2021, khan_empirical_2022}.  Type checking has been made easier by popularity of editors that do it by default. This helps developers find bugs and utilize type information more easily. 

Type-annotated Python allows developers to utilize Pythons core strengths while improving on shortcomings that have historically widely applied to dynamic scripting languages.

Projects should adopt Python type annotations when they want to spend development efforts on improving the maintainability, correctness, and static analysis of their code base. Intentions on coverage, specifically which internal and public APIs should end up covered, should be set  \cite{jin_where_to_start_2021}. Some time should be spent on estimating where correctness and analysis improvements provide most benefit. Often this would be places where type annotation is not exceptionally difficult, and which often result in bugs. When type annotating proves difficult it can sometimes be a hint that program structure is complicated, and could benefit from refactoring.

There are avenues for future research on utilizing Python type annotation most effectively, comparing type checkers, and figuring out what the adoption rate and quality of type hints has been since 2022. Research to use the type annotations for performance and correctness benefits in runtime, and popularity of existing solutions could be useful. It could be possible to compare type annotations to other software quality assurance tools, such as automated testing, and see how optimal effort - benefit tradeoffs can be found. There is also demand for explanations on how to adopt type hints.

% Considering size of Python there seem to be lots of untapped avenues for future research.